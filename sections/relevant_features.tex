\section{Relevant Features} \label{sec:body_relevant_features}
This section elaborates on different features that are potentially useful for the metric. The most prevalent feature in the literature is vividness of language. Other research highlights the effectiveness of bizarre imagery. Finally, I stress the importance of the input words (Radical meaninng + Kanji meaning) in the generated mnemonic.

\subsection{Vividness}
To highlight the importance of vividness consider the following example for the Kanji 驢 that means donkey and consists of the Radicals meaning: horse, tiger, rice paddy and plate. Given each Radical meaning is associated with a concrete image in one's mind, it is rather easy to think of a suitable mnemonic:

\emph{Imagine you are sitting in front of a delightful rice paddy while eating white rice off a white porcelain plate. Suddenly your gaze wanders into the distance where you spot a horse and a tiger racing each other. You chuckle given this odd sight and make bets on who might win. Upon a closer look you see a donkey far behind the two, who appears to be racing as well. He is sweating and his tongue is hanging out. You think, it wouldn't be wise to bet on the donkey}.

In contrast, attempting to create a mnemonic for the Kanji 唯 (solely) with the Radicals meaning turkey and mouth appears to be much more demanding on one's creativity. One mnemonic could be:

\emph{It is Thanksgiving and as you inevitably must expect it is turkey for dinner. You are very excited since you can solely put turkey into your mouth.}

The difference between the two examples is best captured by the word \emph{solely}. All the other words are very concrete, and easily imaginable, leaving not much room for ambiguity. It is not unlikely that one would remember the words \emph{only} or \emph{just} instead of \emph{solely}, as these convey the same meaning. Perhaps one would not remember it at all.

On the other hand take the Kanji 談 (discuss). It's Radical meanings are \emph{fire} and \emph{say}. The words \emph{discuss} and \emph{say} are also rather ambiguous in terms of the mental image they induce. However, given the following mnemonic they become arguably more memorable: 

\emph{If you say things with fire, you are probably discussing something. Imagine that you are almost spitting fire because the discussion is so heated.}

Phrases like "saying things with fire" or "spitting fire while saying something" go beyond the initially rather abstract words and produce very concrete mental images. My first observation here is that vivid language seems to be a good indication of effective mnemonics. This feature of mnemonics has been examined in  several studies \cite{campos_2004} \cite{campos_2011} \cite{kordjazi2014effect}.

In \cite{campos_2011} they found that participants using keyword mnemonics not only outperformed the rote learning group, but also that concrete, vivid words where recalled significantly easier. In the rote group however there was no significant difference between concrete and abstract words.

They explained these findings with Paivio's Dual Coding Theory. With concrete words keyword mnemonics create mental images effortlessly, such that the information is stored in both brain hemispheres. In one as visually and in the other verbally.

Conversely, the rote method leaves out the visual part such that the information is only stored in one hemisphere. Hence there is no striking difference in recall between concrete and abstract words for the rote method. This also explains the better recall for the keyword mnemonic group, even after one day.

The study \cite{campos_2004} yielded similar results although they pointed to a strong recession of recall over longer time intervals in groups using the keyword mnemonics. This highlights that mnemonic devices don't act as a replacement for common learning methods but rather as an important enhancement.

Overall vivid language has been shown to be recalled more easily. Therefore it serves as a good candidate for the metric.
\subsection{Bizarre Imagery}
Returning one more time to the mnemonic examples presented previously, I would like to highlight their peculiarity. It is a very bizarre thing to see the three animals racing each other. Strange or novel experiences catch our attention as they interrupt the flow of explaining the world with what is known to us individually. Therefore these experiences are more memorable. Ample of research suggests that bizarre mental imagery indeed improves recall \cite{bizzare_obrian} \cite{mahdi2018effect}. Moreover, it is possible that one can amplify abstract words by using bizarre imagery. If we modify the previous example to be more bizarre, the Kanji meaning \emph{solely} may be recalled more easily:

\emph{It is Thanksgiving and as you inevitably must expect it is turkey for dinner. One of the guests starts shoving the turkey up his butt. After a moment of shock you head over to that poor fellow and  enlighten him with the fact that one can solely consume turkey with their mouth.}

This example is so bizarre, that it is almost disturbing. The importance of the word \emph{solely} is raised as one can solely consume turkey with their mouth. 

Given these insights, bizarre imagery should be considered as candidate for the metric.

\subsection{Centrality}
Lastly, I want to stress that it is crucial to form strong and meaningful connections between the input words. Consider the mnemonic for \emph{discuss} from earlier:

\emph{If you say things with fire, you are probably discussing something. Imagine that you are almost spitting fire because the discussion is so heated}.

As can be seen the three input words are very dependant on each other. "Saying things with fire" is a rather distinct way of saying something that can easily described as \emph{discussing}.

Compare this example with: \emph{You say "fire" while watching a discussion on television}. While it is not hard to imagine the scene, the connections between the input words are very loose. I hypothesize that this in turn complicates the ease with which the meaning \emph{discuss} can be inferred.  

On a more general note, input words should be well represented and connected meaningfully in the mnemonic. I have chose the term \emph{centrality} to describe this phenomenon.