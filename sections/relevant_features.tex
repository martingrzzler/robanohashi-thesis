\section{Relevant Features} \label{sec:body_relevant_features}
When creating mnemonics it sometimes is very easy to come up with somethings that sticks while other times it proves to be quite difficult.

Let's see an example. 驢 means donkey and consists of the Radicals: horse, tiger, rice paddy and plate. I can only make assumptions, but I imagine it to be quite easy for you to think of a short mnemonic story that combines the words in a meaningful way. E.g. \emph{imagine you are sitting in front of a delightful rice paddy while eating white rice off a white porcelain plate. Suddenly your gaze wanders into the distance where you spot a horse and a tiger racing each other. You chuckle given this odd sight and make bets on who might win. Upon a closer look you see a donkey far behind the two, who appears to be racing as well. He is sweating and his tongue is hanging out. You think, it wouldn't be wise to bet on the donkey}.

In contrast, try to do the same for the following Kanji 唯 (solely) with the Radicals turkey and mouth. While it is still possible to come up with a mnemonic, it seems a lot more difficult. One mnemonic could be: \emph{It is Thanksgiving and as you inevitably must expect it is turkey for dinner. One of the guests starts shoving the turkey up his butt.  After a moment of shock you head over to that poor fellow and  enlighten him with the fact that one can solely consume turkey with their mouth.} Now I hope that this mental image won't disturb you for too long, but quite frankly, this is the point. I hope you agree that the difference between the two examples is best captured by the word \emph{solely}. All the other words are very concrete, and easily imaginable, leaving not much room for ambiguity.

Although it may not be difficult to recall the whole turkey disaster, you may not remember the word solely very well. It is not unlikely that you'd remember it to have been \emph{only} or \emph{just} instead, as these would still be applicable. Perhaps you wouldn't remember it at all.

On the other hand take the Kanji 談 (discuss). It's Radical components are \emph{fire} and \emph{say}. The words discuss and say are also rather ambiguous in terms of the mental image they induce. However, given the following mnemonic: \emph{If you say things with fire, you are probably discussing something. Imagine that you are almost spitting fire because the discussion is so heated}, they become arguably more memorable. Phrases like "saying things with fire" or "spitting fire while saying something" go beyond the initially rather abstract words and produce very concrete mental images. My first observation here is that concrete, imageable language seems to be a good indication of effective mnemonics. This feature of mnemonics has been examined in  several studies \cite{campos_2004} \cite{campos_2011} \cite{kordjazi2014effect}.

In \cite{campos_2011} they found that participants using keyword mnemonics not only outperformed the rote learning group, but also that concrete, vivid words where recalled significantly easier. In the rote group however there was no significant difference between concrete and abstract words. They explained these findings with Paivio's Dual Coding Theory. With concrete words keyword mnemonics create mental images effortlessly, such that the information is stored in both brain hemispheres. In one as visual data and in the other as language. Conversely, the rote method leaves out the visual part such that the information is only stored in one hemisphere. Hence there is no striking difference in recall between concrete and abstract words for the rote method. This also explains the better recall for the keyword mnemonic group, even after one day. The study \cite{campos_2004} yielded similar results although they pointed to a strong recession of recall over longer time intervals in groups using the keyword mnemonics. This highlights again that mnemonic devices don't act as a replacement for common learning methods but rather as an important enhancement. Moreover, they compared different mnemonic generation techniques. Subject, experimenter and peer-generated mnemonics had been compared with the rote learning group. Interestingly, immediate recall was significantly higher for the peer-generated mnemonics compared with all other groups. This further supports the idea of building a platform where mnemonics can be shared with a community. Additionally, they considered that subjects trained in mnemonic techniques won't face difficulties in coming up with mnemonics even for low-vividness words, although they did not test this experimentally. This brings me back to my earlier example where I showed that some words may be rather abstract by themselves, but can be quite vivid assuming one includes them in an appropriate phrase. Given that the peer-generated mnemonics worked very well in the study, I find it plausible to suggest that mnemonics created by language models can be just as effective or possibly even better than subjects trained in the craft. I want to stress here that the research suggests that contrary to what one may think, subject-generated mnemonics aren't more effective and that learners may grossly benefit from high-quality, generated mnemonics, as this mostly saves time. 

Returning one more time to the mnemonic examples I gave earlier I would like to highlight their peculiarity. It is a very bizarre thing to see the three animals racing each other and I think the turkey disaster speaks for itself. Strange or novel experiences catch our attention as they interrupt the flow of explaining the world with what is known to us individually. Therefore these experiences are more memorable. Ample of research suggests that bizarre mental imagery indeed improves recall \cite{bizzare_obrian} \cite{mahdi2018effect}. 

Additionally it may also be worth considering the sentiment of mnemonics since emotionally loaded mental imagery may be more memorable as well. I found however no research on that.

Lastly, I want to stress that it is crucial to form strong and meaningful connections between the words to be remembered. Consider the mnemonic for \emph{discuss} from earlier: \emph{If you say things with fire, you are probably discussing something. Imagine that you are almost spitting fire because the discussion is so heated}. As you hopefully agree the three input words are very dependant on each other. "Saying things with fire" is a rather distinct way of saying something that can easily described as \emph{discussing}. Compare the previous example with: \emph{You say "fire" while watching a discussion on television}. While it is not hard to imagine the scene, the connections between the input words are very loose. I hypothesize that this in turn complicates the ease with which the meaning \emph{discuss} can be inferred.  