\section{Introduction} \label{sec:body_motivation}

The use of mnemonic techniques has been proven effective in enhancing memory and learning \cite{campos_2011, putnam_2015}. Despite their benefits, their application in traditional education remains limited, and their effectiveness can vary depending on individual experiences. The advancements in artificial intelligence (AI) present new opportunities to integrate mnemonic techniques with education. While the creative effort of formulating mnemonics can be time-consuming for untrained learners, generative AI has the potential to alleviate this burden and improve learning efficiency. Inevitably, evaluating the generated mnemonics is crucial to optimizing the generation process.

This thesis aims to investigate the possibility of developing a quantitative metric for evaluating the quality of mnemonics generated using AI and language models, with a particular focus on learning Japanese Kanji.

In the fundamentals (section \ref{sec:fundamentals}) I provide the necessary background of keyword mnemonics in the context of Japanese Kanji learning, the potential of AI to generate them and the unique challenges for evaluating generated mnemonics. Following, I give an overview of existing methods for evaluation of language models in section \ref{sec:body_state_of_the_art}. After this I identify features unique to keyword mnemonics that can be useful for the evaluation metric in section \ref{sec:body_relevant_features}. Section \ref{sec:body_evaluation_data} explains the approach taken to evaluate the previously identified features. Following, the features are evaluated in section \ref{sec:body_feature_evaluation}. Lastly, I discuss my findings and point to future work in section \ref{sec:conclusion}.
