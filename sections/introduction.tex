\section{Introduction} \label{sec:body_motivation}

The use of mnemonic techniques has been proven effective in enhancing memory and learning. Despite their benefits, their application in traditional education remains limited, and their effectiveness can vary depending on individual experiences. The advancements in artificial intelligence (AI) present new opportunities to integrate mnemonic techniques with education. While the creative effort of formulating mnemonics can be time-consuming for untrained learners, generative AI has the potential to alleviate this burden and improve learning efficiency. Inevitably, evaluating the generated mnemonics is crucial to optimizing the generation process.

This paper aims to investigate the possibility of developing a quantitative metric for evaluating the quality of mnemonics generated using AI and language models, with a particular focus on learning Japanese Kanji.

